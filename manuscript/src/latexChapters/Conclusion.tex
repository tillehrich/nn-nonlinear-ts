\section{Conclusion}
\label{chap:conclusion}

In summary it can be said that even though a nonlinear structure was found in the time series, NNs did not perform better at prediction tasks compared to traditional
linear ARIMA models. In view of the interpretability and parsimony of traditional models, one should always stick to simpler models that are easier to interpret, if there
is no sufficient advantage in using more complicated methods. There are many factors that could contribute to the poor results for NNs, for example the search space being too large to explore with
reasonable computing power and the random search therefore missing global minima. Possible enhancements could be the usage of more sophisticated NN architectures, like RNNs with LSTM, that have already succesfully been applied to
forecasting problems \citep{Livieris2020}. This result also shows that NNs are not always the silver bullet they are often made out to be. Therefore in general, more traditional
nonlinear methods like Smooth Transition Models or Kernel Regression should also be considered.

\clearpage