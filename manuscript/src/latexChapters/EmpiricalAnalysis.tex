\section{Empirical Analysis}
\label{chap:empirical}

For the purpose of illustrating the theoretical concepts described in chapters \ref{chap:linear} and \ref{chap:testing},
and to compare the forecasting accuracy of the models in chapter~\ref{chap:benchmark}, monthly data on midwest housing starts in the US from the FRED-MD Database is used~\citep{McCracken2016},
ranging from January 1959 to October 2020.

\begin{figure}[hbt!]
	\centering

	\includegraphics[width=11cm]{latexGraphics/houstmw.png}

	\vspace{-1cm} % Unterschrift näher an die Abbildung
	\caption{Midwest housing starts}
	\label{fig:houstmw}
\end{figure}

The series has no clear up- or downward trending behaviour and seems to wander around non- linearly with a clear breaks in 2009 due to the global recession.
Using the ADF Test with only a constant in the test regression, a p-value of~\adfresult~is obtained indicating stationarity on the 10\% confidence level.
Next, an ARMA model to the data using the automated procedure mentioned in chapter~\ref{subsec:arima}, described in  \citet[chap. 3]{Hyndman2008},
optimizing by AIC and disregarding integration, as stationarity has been shown. After visual investigation of the ACF (figure~\ref{fig:acf_houstmw}) and PACF (figure~\ref{fig:pacf_houstmw}) of the series, the maximum
values of the ARMA parameters are set to the maximum number of significant lags in either the ACF or PACF, neglecting weakly significant correlations that
occur after many lags of insignificant correlations.\\

\begin{figure}[hbt!]
	\centering

	\includegraphics[width=11cm]{latexGraphics/acf_houstmw.png}

	\vspace{-1cm} % Unterschrift näher an die Abbildung
	\caption{ACF for midwest housing starts}
	\label{fig:acf_houstmw}
\end{figure}

\begin{figure}[hbt!]
	\centering

	\includegraphics[width=11cm]{latexGraphics/pacf_houstmw.png}

	\vspace{-1cm} % Unterschrift näher an die Abbildung
	\caption{PACF for midwest housing starts}
	\label{fig:pacf_houstmw}
\end{figure}

\begin{figure}[hbt!]
	\centering

	\includegraphics[width=11cm]{latexGraphics/u_hat_houstmw.png}

	\vspace{-1cm} % Unterschrift näher an die Abbildung
	\caption{Residuals of ARMA model}
	\label{fig:u_hat_houstmw}
\end{figure}

\begin{figure}[hbt!]
	\centering

	\includegraphics[width=11cm]{latexGraphics/acf_u_hat_sq_houstmw.png}

	\vspace{-1cm} % Unterschrift näher an die Abbildung
	\caption{ACF of squared residuals of ARMA model}
	\label{fig:acf_u_hat_sq_houstmw}
\end{figure}

\begin{table}
\centering
\caption{BDS Test Results}
\label{tab:bds}
\begin{tabular}{lllll}
\toprule
m       &        2 &         3 &        4 &        5 \\
p-value &  0.12176 &  0.048708 &  0.07283 &  0.17507 \\
\bottomrule
\end{tabular}
\end{table}


Carrying out the ARCH-LM Test using 3 lags according to figure~\ref{fig:acf_u_hat_sq_houstmw} of the squared Residuals of the selected ARMA(\parima,\qarima)
\footnote{Details are given in tables~\ref{tab:arima1}, \ref{tab:arima2}, \ref{tab:arima3}}, the null hypothesis of no conditional heteroscedasticity for the
residual series of the ARMA model can clearly be rejected (p-value of \archlmresult). \\If a fitted GARCH
model captures the heteroscedasticity present in the residuals, the standardized residuals should exhibit no more heteroscedasticity (see equation~\ref{eq:garch_u}). Hence, the
optimal GARCH model is selected according to the highest p-value of an ARCH-LM test on its standardized residuals, selecting of an array of models with maximum $p$ and
$q$ of 20. This leads to a GARCH(\pgarch, \qgarch) model and a p-value of \archlmresultstd~for the ARCH-LM test on the standardized residuals. This indicates that the present
heteroscedasticty has been cleared from the
residual series. Table~\ref{tab:bds} shows the results of the BDS Test, using $\frac{\epsilon}{\sigma} = 0.5$ as described in \ref{subsec:bds}, for embedding
dimensions 2 to 5. As the p-values for embedding dimensions 3 and 4 are below confidence levels of 5\% and 10\% respectively, the null hypothesis of independence
can be rejected, and nonlinearity can be assumed for the series, as described in chapter~\ref{subsec:bds}. This means that the linear ARMA model is misspecified,
which leads to inconsistent model estimates and poor forecasting capabilities, as described in chapter~\ref{subsec:arima}.

