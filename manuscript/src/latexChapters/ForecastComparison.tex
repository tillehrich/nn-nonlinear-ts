\section{Forecast Comparison}
\label{chap:benchmark}

In this chapter, the forecasting capabilities of ARIMA models, described in chapter~\ref{subsec:arima}, and NNs described in chapter~\ref{chap:neural} will be compared based on the time series analyzed
in chapter~\ref{chap:empirical}, for which nonlinearity can be assumed. The dataset is split into a training and a validation sample. Next, the optimal models are determined
based on the training set and $h$-step out of sample forecasts are computed for each values of the validation sample, as described in equations~\ref{eq:hstep0}
to \ref{eq:hstep3}. $y_t$ is the last value of the training set and real values for $y$ are used, as long as they are part of the training sample.
The forecast accuracy is then compared according to the MSE of each model described in equation~\ref{eq:mse}.

\begin{equation} \label{eq:hstep0}
	y_t = f(y_{t-1},...,y_{t-p})
\end{equation}

\begin{equation} \label{eq:hstep1}
    \hat{y}_{t+1} = f(y_t,...,y_{t-p+1})
\end{equation}

\begin{equation} \label{eq:hstep2}
    \hat{y}_{t+2} = f(\hat{y}_{t+1},y_t,...,y_{t-p+2})~\text{...}
\end{equation}

\begin{equation} \label{eq:hstep3}
    \hat{y}_{t+h} = f(\hat{y}_{t+h-1},\hat{y}_{t+h-2},...,\hat{y}_{t+h-p})
\end{equation}

\begin{equation} \label{eq:mse}
    MSE = \frac{1}{H}\sum_{h=1}^{H}(y_{t+h}-\hat{y}_{t+h})^2
\end{equation}

The optimal specification of the ARIMA model is determined as in chapter~\ref{chap:empirical}. The optimal specification of the NN is determined via randomized grid
search, allowing for nodesizes from 1 to 100 in increments of 5, maximum number of hidden layers ranging from 1 to 5 and possible values for alpha being generated
on a log scale from 0.001 to 0.05. Furthermore, the relu activation function described in equation~\ref{eq:relu} was used for all hidden layers due to best performance
among the three options described in chapter~\ref{subsec:act}. For fitting the model, the Adam algorithm \citep{Kingma2017} was used, which is a variant of the backpropagation
algorithm described in ~\ref{subsec:fitting}. Out of the grid spanned by all parameter combinations, 2000 combinations are picked at random and trained based on
of the first 90\% of the training sample, using the remaining 10\% as test set to evaluate the out of sample MSE. The best model is then picked, and retrained on the
entire training set.\\
In order to get a more consistent measure for the forecasting accuracy, this whole procedure is repeated for 50 subsets of the original series, each time eliminating
one more observation from the series, starting with the most recent one. Forecasts are computed for horizons 1, 5, and 10 in order to compare short- to long term forecasting
capabilitiy. In the end, models are compared according to the mean of their MSE over all subsamples per forecast horizon.

\begin{table}
\centering
\caption{Mean MSE for different forecast horizons}
\label{tab:mse}
\begin{tabular}{lrr}
\toprule
 h &     MLP &   ARMA \\
\midrule
 1 &  979.43 & 819.13 \\
 5 &  984.88 & 997.30 \\
10 & 1016.44 & 968.13 \\
\bottomrule
\end{tabular}
\end{table}


\begin{figure}
	\centering


	\includegraphics[width=11cm]{latexGraphics/benchmark_5_ahead.png}

	\vspace{-0.5cm} % Unterschrift näher an die Abbildung
	\caption{MSE for all subsamples and horizon = 5}
	\label{fig:benchmark5}
\end{figure}

Table~\ref{tab:mse} shows that for horizons 1 and 10, the ARMA model exhibited better forecasting capability, while for horizon 5, the MLP showed better performance. In general,
the accuracies are close to another, therefore no real dominance of one method over the other can be established. Figure~\ref{fig:benchmark5} also shows that predictive performance varies a lot
depending on the subsample, with both models achiving better results than the other, but
again no dominance can be inferred even for subsamples of the data. Figure~\ref{fig:benchmark} supports this claim. Furthermore, tables~\ref{tab:specifications1},~\ref{tab:specifications5} and
\ref{tab:specifications10} in the Appendix show that the structure of the NN changes drastically over subsamples. Alltogether this indicates, that neither of the model types
are able to capture the underlying DGP accurately.