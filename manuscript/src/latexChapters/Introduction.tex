\section{Introduction}
\label{chap:introduction}

Modern time series analysis is characterized by an abundance of both data and possible modeling algorithms used for inference assessment and prediction.
The latter range from traditional statistical approaches like ARIMA, GARCH or Exponential Smoothing to modern Machine Learning methods like Penalized Regression,
Regression Trees or Deep Learning. Brief overviews over available methods as well as empirical comparisons are given in~\cite{Ahmed2010, Makridakis2018, Hyndman2020}.
Machine Learning methods have gained popularity in recent years~\citep[p. 1]{Makridakis2018}~and appeal through their ability to map complex functional forms without
imposing restrictions that are inherent to traditional statistical methods \citep[p. 1083]{Hill1996}, but also show their limitations in their
Black-Box nature \citep[p. 351, 352]{Hastie2009} and mixed success in empirical comparisons \citep[p. 594]{Ahmed2010},\citep[p. 1]{Makridakis2018}.
Neural Networks as representatives of Deep Learning have gained particular popularity, with recent econometric applications in \cite{Bucci2019, Sestanovic2021}.\\
Aimed at providing an intuition about the forecasting accuracy of Neural Networks for nonlinear time series, as well as laying out the underlying theoretical
groundwork, the rest of this thesis is structured as follows:\\
Chapter~\ref{chap:linear} gives a brief intuition about linear time series models and relevant assumptions. Chapter~\ref{chap:testing} elaborates on the subject of
nonlinearity in time series and designated test procedures. Chapter~\ref{chap:neural} gives a brief introduction to Neural Networks. Chapter~\ref{chap:empirical}
analyses U.S. housing data with emphasis on detecting nonlinearity in the series. Chapter~\ref{chap:benchmark} compares the forecasting accuracy of ARIMA and NN models,
chapter~\ref{chap:conclusion} sums up the results.