\section{Linear Time Series Modeling}
\label{chap:linear}


Linear models still lay the groundwork of modern time series analysis today. They are the first resort for any time series practitioner and the benchmark
for all alternative modeling approaches.
This chapter is based on \cite{Luetkepohl2004}, who give a detailed introduction into econometric time series analysis.
Given the fact that an extensive overview is out of the scope of this thesis, only the most important concepts will be revised.

\subsection{Stationarity}
\label{subsec:stationarity}
The basic assumption of stationarity involves two conditions of a time series process $\{y_t\}_{t \in T}$, constant mean of all members of a stationary process,
and time invariance of the variance of the process:

\begin{equation} \label{eq:stationarity_mean}
	E(y_t) = \mu_y \forall~t \in T
\end{equation}

\begin{equation} \label{eq:stationarity_variance}
	E[(y_t-\mu_y)(y_{t-h}-\mu_y)] = \gamma_h ~{\forall~t\in T}~\text{and all integers $h$, $(t-h) \in T$ }
\end{equation}

Failure to account for the absence of these properties results in biased model estimates, and model parameters can't be tested using traditional tests. Also the
model will exhibit poor predictive performance. An adequate test for the presence of non-stationarity is the ADF Test, described in
\citep[chap. 2.7.1]{Luetkepohl2004}. Here it is important to point out that the test equation of the ADF test is not ideally equipped to handle nonlinear DGPs,
as the test equation has a linear structure. However, it has been shown that the ADF test still has reasonable power against nonlinear alternatives
\citep[p. 40]{Demetrescu2013}, \citep[p. 54]{Corradi2000} and might be used even when nonlinearity is present. \citet{Demetrescu2013,Corradi2000} also propose and
discuss alternative tests with nonlinear alternatives, which lay outside the scope of this thesis.

\subsection{ARIMA Models}
\label{subsec:arima}
ARIMA models have been the workhorse of linear time series models for a long time. They will serve as the baseline for the comparison of forecasts between linear and
nonlinear methods. The methodology briefly described here follows \citet[chap. 2.3.3]{Luetkepohl2004}.\\
A general ARMA$(p,q)$ model can be written as follows:
\begin{equation} \label{eq:arma}
	y_t = \alpha_1 y_{t-1}+...+\alpha_p y_{t-p}+u_t+m_1 u_{t-1}+...+m_q u_{t-q}
\end{equation}
with $u_t \sim i.i.d~WN(0,\sigma^2)$. This assumption is also called the \emph{i.i.d} assumption, which will be subject of the diagnostic tests discussed in
chapter~\ref{chap:testing}, as violations of this assumption lead to inconsistent model estimates and poor forecasting capabilities. This model is integrated by
oder $d$, if differencing the data $d$ times is required in order to achieve stationarity, which yields the general ARIMA$(p,d,q)$ model. There is a multitude of procedures at hand
to determine the optimal orders of the model, with one of the most popular being described in \citet[chap. 3]{Hyndman2008}. It is most feasible in this use case as it is
tried and tested, and can be implemented automatically.
